% Options for packages loaded elsewhere
\PassOptionsToPackage{unicode}{hyperref}
\PassOptionsToPackage{hyphens}{url}
%
\documentclass[
]{article}
\usepackage{amsmath,amssymb}
\usepackage{iftex}
\ifPDFTeX
  \usepackage[T1]{fontenc}
  \usepackage[utf8]{inputenc}
  \usepackage{textcomp} % provide euro and other symbols
\else % if luatex or xetex
  \usepackage{unicode-math} % this also loads fontspec
  \defaultfontfeatures{Scale=MatchLowercase}
  \defaultfontfeatures[\rmfamily]{Ligatures=TeX,Scale=1}
\fi
\usepackage{lmodern}
\ifPDFTeX\else
  % xetex/luatex font selection
\fi
% Use upquote if available, for straight quotes in verbatim environments
\IfFileExists{upquote.sty}{\usepackage{upquote}}{}
\IfFileExists{microtype.sty}{% use microtype if available
  \usepackage[]{microtype}
  \UseMicrotypeSet[protrusion]{basicmath} % disable protrusion for tt fonts
}{}
\makeatletter
\@ifundefined{KOMAClassName}{% if non-KOMA class
  \IfFileExists{parskip.sty}{%
    \usepackage{parskip}
  }{% else
    \setlength{\parindent}{0pt}
    \setlength{\parskip}{6pt plus 2pt minus 1pt}}
}{% if KOMA class
  \KOMAoptions{parskip=half}}
\makeatother
\usepackage{xcolor}
\usepackage[margin=1in]{geometry}
\usepackage{color}
\usepackage{fancyvrb}
\newcommand{\VerbBar}{|}
\newcommand{\VERB}{\Verb[commandchars=\\\{\}]}
\DefineVerbatimEnvironment{Highlighting}{Verbatim}{commandchars=\\\{\}}
% Add ',fontsize=\small' for more characters per line
\usepackage{framed}
\definecolor{shadecolor}{RGB}{248,248,248}
\newenvironment{Shaded}{\begin{snugshade}}{\end{snugshade}}
\newcommand{\AlertTok}[1]{\textcolor[rgb]{0.94,0.16,0.16}{#1}}
\newcommand{\AnnotationTok}[1]{\textcolor[rgb]{0.56,0.35,0.01}{\textbf{\textit{#1}}}}
\newcommand{\AttributeTok}[1]{\textcolor[rgb]{0.13,0.29,0.53}{#1}}
\newcommand{\BaseNTok}[1]{\textcolor[rgb]{0.00,0.00,0.81}{#1}}
\newcommand{\BuiltInTok}[1]{#1}
\newcommand{\CharTok}[1]{\textcolor[rgb]{0.31,0.60,0.02}{#1}}
\newcommand{\CommentTok}[1]{\textcolor[rgb]{0.56,0.35,0.01}{\textit{#1}}}
\newcommand{\CommentVarTok}[1]{\textcolor[rgb]{0.56,0.35,0.01}{\textbf{\textit{#1}}}}
\newcommand{\ConstantTok}[1]{\textcolor[rgb]{0.56,0.35,0.01}{#1}}
\newcommand{\ControlFlowTok}[1]{\textcolor[rgb]{0.13,0.29,0.53}{\textbf{#1}}}
\newcommand{\DataTypeTok}[1]{\textcolor[rgb]{0.13,0.29,0.53}{#1}}
\newcommand{\DecValTok}[1]{\textcolor[rgb]{0.00,0.00,0.81}{#1}}
\newcommand{\DocumentationTok}[1]{\textcolor[rgb]{0.56,0.35,0.01}{\textbf{\textit{#1}}}}
\newcommand{\ErrorTok}[1]{\textcolor[rgb]{0.64,0.00,0.00}{\textbf{#1}}}
\newcommand{\ExtensionTok}[1]{#1}
\newcommand{\FloatTok}[1]{\textcolor[rgb]{0.00,0.00,0.81}{#1}}
\newcommand{\FunctionTok}[1]{\textcolor[rgb]{0.13,0.29,0.53}{\textbf{#1}}}
\newcommand{\ImportTok}[1]{#1}
\newcommand{\InformationTok}[1]{\textcolor[rgb]{0.56,0.35,0.01}{\textbf{\textit{#1}}}}
\newcommand{\KeywordTok}[1]{\textcolor[rgb]{0.13,0.29,0.53}{\textbf{#1}}}
\newcommand{\NormalTok}[1]{#1}
\newcommand{\OperatorTok}[1]{\textcolor[rgb]{0.81,0.36,0.00}{\textbf{#1}}}
\newcommand{\OtherTok}[1]{\textcolor[rgb]{0.56,0.35,0.01}{#1}}
\newcommand{\PreprocessorTok}[1]{\textcolor[rgb]{0.56,0.35,0.01}{\textit{#1}}}
\newcommand{\RegionMarkerTok}[1]{#1}
\newcommand{\SpecialCharTok}[1]{\textcolor[rgb]{0.81,0.36,0.00}{\textbf{#1}}}
\newcommand{\SpecialStringTok}[1]{\textcolor[rgb]{0.31,0.60,0.02}{#1}}
\newcommand{\StringTok}[1]{\textcolor[rgb]{0.31,0.60,0.02}{#1}}
\newcommand{\VariableTok}[1]{\textcolor[rgb]{0.00,0.00,0.00}{#1}}
\newcommand{\VerbatimStringTok}[1]{\textcolor[rgb]{0.31,0.60,0.02}{#1}}
\newcommand{\WarningTok}[1]{\textcolor[rgb]{0.56,0.35,0.01}{\textbf{\textit{#1}}}}
\usepackage{graphicx}
\makeatletter
\def\maxwidth{\ifdim\Gin@nat@width>\linewidth\linewidth\else\Gin@nat@width\fi}
\def\maxheight{\ifdim\Gin@nat@height>\textheight\textheight\else\Gin@nat@height\fi}
\makeatother
% Scale images if necessary, so that they will not overflow the page
% margins by default, and it is still possible to overwrite the defaults
% using explicit options in \includegraphics[width, height, ...]{}
\setkeys{Gin}{width=\maxwidth,height=\maxheight,keepaspectratio}
% Set default figure placement to htbp
\makeatletter
\def\fps@figure{htbp}
\makeatother
\setlength{\emergencystretch}{3em} % prevent overfull lines
\providecommand{\tightlist}{%
  \setlength{\itemsep}{0pt}\setlength{\parskip}{0pt}}
\setcounter{secnumdepth}{-\maxdimen} % remove section numbering
\ifLuaTeX
  \usepackage{selnolig}  % disable illegal ligatures
\fi
\usepackage{bookmark}
\IfFileExists{xurl.sty}{\usepackage{xurl}}{} % add URL line breaks if available
\urlstyle{same}
\hypersetup{
  pdftitle={Trabajo Práctico 2},
  pdfauthor={Olivia Luchetti y Martina Mariño},
  hidelinks,
  pdfcreator={LaTeX via pandoc}}

\title{Trabajo Práctico 2}
\author{Olivia Luchetti y Martina Mariño}
\date{}

\begin{document}
\maketitle

\begin{Shaded}
\begin{Highlighting}[]
\FunctionTok{set.seed}\NormalTok{(}\DecValTok{0264}\NormalTok{)}
\end{Highlighting}
\end{Shaded}

\subsection{Ejercicio 1 - Distribución Uniforme
(0,18)}\label{ejercicio-1---distribuciuxf3n-uniforme-018}

\subsubsection{Ejercicio 1.a: Función para Generar
Realizaciones}\label{ejercicio-1.a-funciuxf3n-para-generar-realizaciones}

Creamos una función que devuelva un vector con R realizaciones de una
variable aleatoria \(X - U(0,18)\).

\begin{Shaded}
\begin{Highlighting}[]
\NormalTok{X\_dist }\OtherTok{\textless{}{-}} \ControlFlowTok{function}\NormalTok{(R) \{}
  \FunctionTok{return}\NormalTok{(}\FunctionTok{runif}\NormalTok{(}\AttributeTok{n =}\NormalTok{ R, }\AttributeTok{min =} \DecValTok{0}\NormalTok{, }\AttributeTok{max =} \DecValTok{18}\NormalTok{))}
\NormalTok{\}}
\end{Highlighting}
\end{Shaded}

\subsubsection{Ejercicio 1.b: Cálculo de Media y Varianza
Muestral}\label{ejercicio-1.b-cuxe1lculo-de-media-y-varianza-muestral}

Calculamos la media y la varianza muestral para diferentes tamaños de
muestra R.

\begin{Shaded}
\begin{Highlighting}[]
\NormalTok{comparacion }\OtherTok{\textless{}{-}} \FunctionTok{data.frame}\NormalTok{(}
  \AttributeTok{R =} \FunctionTok{c}\NormalTok{(}\DecValTok{2}\NormalTok{, }\DecValTok{30}\NormalTok{, }\DecValTok{100}\NormalTok{, }\DecValTok{10}\SpecialCharTok{\^{}}\DecValTok{4}\NormalTok{), }
    \AttributeTok{Media\_muestral =} \FunctionTok{c}\NormalTok{(}
    \FunctionTok{mean}\NormalTok{(}\FunctionTok{X\_dist}\NormalTok{(}\DecValTok{2}\NormalTok{)),}
    \FunctionTok{mean}\NormalTok{(}\FunctionTok{X\_dist}\NormalTok{(}\DecValTok{30}\NormalTok{)),}
    \FunctionTok{mean}\NormalTok{(}\FunctionTok{X\_dist}\NormalTok{(}\DecValTok{100}\NormalTok{)),}
    \FunctionTok{mean}\NormalTok{(}\FunctionTok{X\_dist}\NormalTok{(}\DecValTok{10}\SpecialCharTok{\^{}}\DecValTok{4}\NormalTok{))}
\NormalTok{  ),}
  \AttributeTok{Varianza\_muestral =} \FunctionTok{c}\NormalTok{(}
    \FunctionTok{var}\NormalTok{(}\FunctionTok{X\_dist}\NormalTok{(}\DecValTok{2}\NormalTok{)),}
    \FunctionTok{var}\NormalTok{(}\FunctionTok{X\_dist}\NormalTok{(}\DecValTok{30}\NormalTok{)),}
    \FunctionTok{var}\NormalTok{(}\FunctionTok{X\_dist}\NormalTok{(}\DecValTok{100}\NormalTok{)),}
    \FunctionTok{var}\NormalTok{(}\FunctionTok{X\_dist}\NormalTok{(}\DecValTok{10}\SpecialCharTok{\^{}}\DecValTok{4}\NormalTok{))}
\NormalTok{  )}
\NormalTok{)}

\FunctionTok{print}\NormalTok{(comparacion)}
\end{Highlighting}
\end{Shaded}

\begin{verbatim}
##       R Media_muestral Varianza_muestral
## 1     2      12.783315          43.82488
## 2    30       8.589811          25.11464
## 3   100       9.191024          26.50414
## 4 10000       8.937333          27.00084
\end{verbatim}

\subsubsection{Ejercicio 1.c: Valores Teóricos y
Comparación}\label{ejercicio-1.c-valores-teuxf3ricos-y-comparaciuxf3n}

El valor teórico de la esperanza \(E(X)\) y la varianza \(V(X)\) para
una distribución uniforme \(U(0,18)\) son:

\begin{itemize}
\item
  \(E(X) = (a + b) / 2 = (0 + 18) / 2 = 9\)
\item
  \(V(X) = (b - a)^2 / 12 = (18 − 0)^2 / 12 = 27\)
\end{itemize}

\paragraph{Comparación de Valores Teóricos y
Muestrales:}\label{comparaciuxf3n-de-valores-teuxf3ricos-y-muestrales}

\begin{Shaded}
\begin{Highlighting}[]
\NormalTok{E\_X }\OtherTok{=} \DecValTok{9}
\NormalTok{Var\_X }\OtherTok{=} \DecValTok{27}

\NormalTok{R\_values }\OtherTok{\textless{}{-}} \FunctionTok{c}\NormalTok{(}\DecValTok{2}\NormalTok{, }\DecValTok{30}\NormalTok{, }\DecValTok{100}\NormalTok{, }\DecValTok{10000}\NormalTok{)}

\NormalTok{X }\OtherTok{\textless{}{-}} \FunctionTok{numeric}\NormalTok{(}\DecValTok{4}\NormalTok{)}
\NormalTok{s }\OtherTok{\textless{}{-}} \FunctionTok{numeric}\NormalTok{(}\DecValTok{4}\NormalTok{)}
\NormalTok{abs\_X\_E\_X }\OtherTok{\textless{}{-}} \FunctionTok{numeric}\NormalTok{(}\DecValTok{4}\NormalTok{)}
\NormalTok{abs\_s2\_VarX }\OtherTok{\textless{}{-}} \FunctionTok{numeric}\NormalTok{(}\DecValTok{4}\NormalTok{)}

\ControlFlowTok{for}\NormalTok{ (i }\ControlFlowTok{in} \DecValTok{1}\SpecialCharTok{:}\DecValTok{4}\NormalTok{) \{}
\NormalTok{  R }\OtherTok{\textless{}{-}}\NormalTok{ R\_values[i]}
\NormalTok{  sample\_X }\OtherTok{\textless{}{-}} \FunctionTok{runif}\NormalTok{(R, }\DecValTok{0}\NormalTok{, }\DecValTok{18}\NormalTok{)}
  
\NormalTok{  X[i] }\OtherTok{\textless{}{-}} \FunctionTok{mean}\NormalTok{(sample\_X)}
\NormalTok{  s[i] }\OtherTok{\textless{}{-}} \FunctionTok{var}\NormalTok{(sample\_X)}
\NormalTok{  abs\_X\_E\_X[i] }\OtherTok{\textless{}{-}} \FunctionTok{abs}\NormalTok{(X[i] }\SpecialCharTok{{-}}\NormalTok{ E\_X)}
\NormalTok{  abs\_s2\_VarX[i] }\OtherTok{\textless{}{-}} \FunctionTok{abs}\NormalTok{(s[i] }\SpecialCharTok{{-}}\NormalTok{ Var\_X)}
\NormalTok{\}}

\NormalTok{R\_values }\OtherTok{\textless{}{-}} \FunctionTok{c}\NormalTok{(R\_values, }\StringTok{"∞"}\NormalTok{)}
\NormalTok{X }\OtherTok{\textless{}{-}} \FunctionTok{c}\NormalTok{(X, E\_X)}
\NormalTok{s }\OtherTok{\textless{}{-}} \FunctionTok{c}\NormalTok{(s, Var\_X)}
\NormalTok{abs\_X\_E\_X }\OtherTok{\textless{}{-}} \FunctionTok{c}\NormalTok{(abs\_X\_E\_X, }\DecValTok{0}\NormalTok{)}
\NormalTok{abs\_s2\_VarX }\OtherTok{\textless{}{-}} \FunctionTok{c}\NormalTok{(abs\_s2\_VarX, }\DecValTok{0}\NormalTok{)}

\NormalTok{tabla }\OtherTok{\textless{}{-}} \FunctionTok{data.frame}\NormalTok{(}
  \AttributeTok{R =}\NormalTok{ R\_values,}
  \AttributeTok{X\_barra =} \FunctionTok{round}\NormalTok{(X, }\DecValTok{3}\NormalTok{),}
  \AttributeTok{s\_2 =} \FunctionTok{round}\NormalTok{(s, }\DecValTok{3}\NormalTok{),}
  \AttributeTok{abs\_X\_barra\_E\_X =} \FunctionTok{round}\NormalTok{(abs\_X\_E\_X, }\DecValTok{3}\NormalTok{),}
  \AttributeTok{abs\_s\_2\_Var\_X =} \FunctionTok{round}\NormalTok{(abs\_s2\_VarX, }\DecValTok{3}\NormalTok{)}
\NormalTok{)}

\FunctionTok{print}\NormalTok{(tabla)}
\end{Highlighting}
\end{Shaded}

\begin{verbatim}
##       R X_barra    s_2 abs_X_barra_E_X abs_s_2_Var_X
## 1     2  12.310 34.557           3.310         7.557
## 2    30   8.854 20.261           0.146         6.739
## 3   100   8.357 24.100           0.643         2.900
## 4 10000   8.968 27.320           0.032         0.320
## 5     ∞   9.000 27.000           0.000         0.000
\end{verbatim}

\paragraph{Conclusión:}\label{conclusiuxf3n}

A medida que el tamaño de la muestra R aumenta, la media y varianza
muestrales se aproximan progresivamente a los valores teóricos. Esto es
consistente con la Ley de los Grandes Números, que establece que, a
medida que aumenta el tamaño de la muestra, las estimaciones muestrales
tienden a converger hacia los parámetros poblacionales. La Ley de los
Grandes Númetos asegura que las estimaciones obtenidas de una muestra
suficientemente grande serán cercanas a los verdaderos valores de la
población de donde provienen. En el contexto de la tabla, se puede
observar cómo, para tamaños de muestra pequeños (como R = 2), la
diferencia entre las estimaciones muestrales y los valores teóricos es
considerable. Sin embargo, al aumentar R, estas diferencias disminuyen,
lo que indica que la media y la varianza muestrales están cada vez más
alineadas con sus valores teóricos. Este comportamiento refleja cómo el
aumento en el tamaño de la muestra no solo mejora la precisión de las
estimaciones, sino que también reduce la variabilidad inherente en las
muestras.

\subsubsection{Ejercicio 1.d: Histogramas de
X}\label{ejercicio-1.d-histogramas-de-x}

Creamos histogramas para R = 100 y R = 10\^{}4 utilizando 30 bines.

\paragraph{Distribución esperada:}\label{distribuciuxf3n-esperada}

Con una distribución uniforme U(0,18), esperamos que el histograma
muestre una distribución bastante uniforme, especialmente para tamaños
de muestra grandes. Con R=100, el histograma debería comenzar a
parecerse a la distribución uniforme. Para R = 10\^{}4, la distribución
debería ser aún más uniforme y parecerse a la distribución teórica.

\includegraphics{TP2_files/figure-latex/unnamed-chunk-5-1.pdf}
\includegraphics{TP2_files/figure-latex/unnamed-chunk-5-2.pdf}

\paragraph{Discusión:}\label{discusiuxf3n}

Al aumentar el tamaño de la muestra, el histograma se acerca más a la
forma esperada de la distribución uniforme. Para muestras pequeñas, el
histograma puede mostrar variabilidad significativa debido a la menor
cantidad de datos. A medida que R aumenta, la variabilidad disminuye y
el histograma se asemeja más a la distribución uniforme teórica.

\subsection{Ejercicio 2 - Promedio de Quince Valores
Independientes}\label{ejercicio-2---promedio-de-quince-valores-independientes}

\subsubsection{Ejercicio 2.a: Función para Generar Realizaciones de
Y}\label{ejercicio-2.a-funciuxf3n-para-generar-realizaciones-de-y}

Creamos una función que devuelva un vector con R realizaciones de Y,
donde cada Y es el promedio de 15 valores independientes de X.

\begin{Shaded}
\begin{Highlighting}[]
\NormalTok{Y\_dist }\OtherTok{\textless{}{-}} \ControlFlowTok{function}\NormalTok{(R) \{}
  \CommentTok{\# Inicializa un vector numérico de longitud R, donde se almacenarán los promedios}
  \CommentTok{\#calculados.}
\NormalTok{  Y }\OtherTok{\textless{}{-}} \FunctionTok{numeric}\NormalTok{(R)}
  \CommentTok{\# Bucle que se repite R veces, una por cada realización de Y.}
  \ControlFlowTok{for}\NormalTok{(i }\ControlFlowTok{in} \DecValTok{1}\SpecialCharTok{:}\NormalTok{R) \{}
    \CommentTok{\# Genera 15 valores independientes de X, donde X sigue una distribución}
    \CommentTok{\#uniforme entre 0 y 18.}
\NormalTok{    X\_15 }\OtherTok{\textless{}{-}} \FunctionTok{runif}\NormalTok{(}\AttributeTok{n =} \DecValTok{15}\NormalTok{, }\AttributeTok{min =} \DecValTok{0}\NormalTok{, }\AttributeTok{max =} \DecValTok{18}\NormalTok{)}
    \CommentTok{\# Calcula el promedio de los 15 valores generados de X y lo almacena en la }
    \CommentTok{\#i{-}ésima posición del vector Y.}
\NormalTok{    Y[i] }\OtherTok{\textless{}{-}} \FunctionTok{mean}\NormalTok{(X\_15)}
\NormalTok{  \}}
  \CommentTok{\# Devuelve el vector Y, que contiene R promedios de 15 valores independientes de}
  \CommentTok{\#X.}
  \FunctionTok{return}\NormalTok{(Y)}
\NormalTok{\}}
\end{Highlighting}
\end{Shaded}

\subsubsection{Ejercicio 2.b: Cálculo de Media y Varianza
Muestral}\label{ejercicio-2.b-cuxe1lculo-de-media-y-varianza-muestral}

Calculamos la media y varianza muestral para diferentes tamaños de
muestra R.

\begin{Shaded}
\begin{Highlighting}[]
\NormalTok{comparacion }\OtherTok{\textless{}{-}} \FunctionTok{data.frame}\NormalTok{(}
  \AttributeTok{R =} \FunctionTok{c}\NormalTok{(}\DecValTok{2}\NormalTok{, }\DecValTok{30}\NormalTok{, }\DecValTok{100}\NormalTok{, }\DecValTok{10}\SpecialCharTok{\^{}}\DecValTok{4}\NormalTok{), }
    \AttributeTok{Media\_muestral =} \FunctionTok{c}\NormalTok{(}
    \FunctionTok{mean}\NormalTok{(}\FunctionTok{Y\_dist}\NormalTok{(}\DecValTok{2}\NormalTok{)),}
    \FunctionTok{mean}\NormalTok{(}\FunctionTok{Y\_dist}\NormalTok{(}\DecValTok{30}\NormalTok{)),}
    \FunctionTok{mean}\NormalTok{(}\FunctionTok{Y\_dist}\NormalTok{(}\DecValTok{100}\NormalTok{)),}
    \FunctionTok{mean}\NormalTok{(}\FunctionTok{Y\_dist}\NormalTok{(}\DecValTok{10}\SpecialCharTok{\^{}}\DecValTok{4}\NormalTok{))}
\NormalTok{  ),}
  \AttributeTok{Varianza\_muestral =} \FunctionTok{c}\NormalTok{(}
    \FunctionTok{var}\NormalTok{(}\FunctionTok{Y\_dist}\NormalTok{(}\DecValTok{2}\NormalTok{)),}
    \FunctionTok{var}\NormalTok{(}\FunctionTok{Y\_dist}\NormalTok{(}\DecValTok{30}\NormalTok{)),}
    \FunctionTok{var}\NormalTok{(}\FunctionTok{Y\_dist}\NormalTok{(}\DecValTok{100}\NormalTok{)),}
    \FunctionTok{var}\NormalTok{(}\FunctionTok{Y\_dist}\NormalTok{(}\DecValTok{10}\SpecialCharTok{\^{}}\DecValTok{4}\NormalTok{))}
\NormalTok{  )}
\NormalTok{)}

\FunctionTok{print}\NormalTok{(comparacion)}
\end{Highlighting}
\end{Shaded}

\begin{verbatim}
##       R Media_muestral Varianza_muestral
## 1     2       9.457222       0.001930543
## 2    30       8.852968       1.886629933
## 3   100       9.268562       1.521618640
## 4 10000       8.998140       1.850213107
\end{verbatim}

\subsubsection{Ejercicio 2.c: Comparación de Medias
Empíricas}\label{ejercicio-2.c-comparaciuxf3n-de-medias-empuxedricas}

Dado que \(E(Y)\) = \(E(X)\), el valor teórico de la esperanza \(E(Y)\)
y la varianza \(V(Y)\) son:

\begin{itemize}
\item
  \(E(Y) = E(X) = 9\)
\item
  \(V(Y) = [(b - a)^2 / 12] / n = V(X)/n = V(X)/15 = 27/15 = 1.8\)
\end{itemize}

\paragraph{Media Esperada y Varianza para
Y:}\label{media-esperada-y-varianza-para-y}

\begin{Shaded}
\begin{Highlighting}[]
\NormalTok{E\_Y\_teorico }\OtherTok{\textless{}{-}} \DecValTok{9}
\NormalTok{V\_Y\_teorico }\OtherTok{\textless{}{-}} \DecValTok{27} \SpecialCharTok{/} \DecValTok{15}

\NormalTok{comparacion }\OtherTok{\textless{}{-}} \FunctionTok{data.frame}\NormalTok{(}
  \AttributeTok{R =} \FunctionTok{c}\NormalTok{(}\DecValTok{2}\NormalTok{, }\DecValTok{30}\NormalTok{, }\DecValTok{100}\NormalTok{, }\DecValTok{10}\SpecialCharTok{\^{}}\DecValTok{4}\NormalTok{), }
    \AttributeTok{Media\_muestral =} \FunctionTok{c}\NormalTok{(}
    \FunctionTok{mean}\NormalTok{(}\FunctionTok{X\_dist}\NormalTok{(}\DecValTok{2}\NormalTok{)),}
    \FunctionTok{mean}\NormalTok{(}\FunctionTok{X\_dist}\NormalTok{(}\DecValTok{30}\NormalTok{)),}
    \FunctionTok{mean}\NormalTok{(}\FunctionTok{X\_dist}\NormalTok{(}\DecValTok{100}\NormalTok{)),}
    \FunctionTok{mean}\NormalTok{(}\FunctionTok{X\_dist}\NormalTok{(}\DecValTok{10}\SpecialCharTok{\^{}}\DecValTok{4}\NormalTok{))}
\NormalTok{  ),}
  \AttributeTok{Varianza\_muestral =} \FunctionTok{c}\NormalTok{(}
    \FunctionTok{var}\NormalTok{(}\FunctionTok{X\_dist}\NormalTok{(}\DecValTok{2}\NormalTok{)),}
    \FunctionTok{var}\NormalTok{(}\FunctionTok{X\_dist}\NormalTok{(}\DecValTok{30}\NormalTok{)),}
    \FunctionTok{var}\NormalTok{(}\FunctionTok{X\_dist}\NormalTok{(}\DecValTok{100}\NormalTok{)),}
    \FunctionTok{var}\NormalTok{(}\FunctionTok{X\_dist}\NormalTok{(}\DecValTok{10}\SpecialCharTok{\^{}}\DecValTok{4}\NormalTok{))}
\NormalTok{  ),}
  \AttributeTok{Media\_teorica =}\NormalTok{ E\_Y\_teorico,}
  \AttributeTok{Varianza\_teorica =}\NormalTok{ V\_Y\_teorico}
\NormalTok{)}

\FunctionTok{print}\NormalTok{(comparacion)}
\end{Highlighting}
\end{Shaded}

\begin{verbatim}
##       R Media_muestral Varianza_muestral Media_teorica Varianza_teorica
## 1     2      10.941158          4.175705             9              1.8
## 2    30       7.774725         21.974596             9              1.8
## 3   100       8.641142         27.634614             9              1.8
## 4 10000       9.011387         27.073208             9              1.8
\end{verbatim}

\paragraph{Discusión:}\label{discusiuxf3n-1}

Las medias empíricas de Y deberían acercarse más al valor esperado
\(E(Y)=9\) comparado con las medias empíricas de X, ya que Y es un
promedio de varias observaciones de X. La Ley de los Grandes Números
garantiza que a medida que el tamaño de la muestra aumenta, la media
empírica se aproxima más al valor esperado de la distribución. Esta ley
asegura que el promedio de una gran cantidad de observaciones será más
preciso en reflejar la media teórica. La varianza de Y debería ser más
pequeña que la varianza de X debido al promedio de múltiples
observaciones.

\subsubsection{Ejercicio 2.d: Histogramas de
Y}\label{ejercicio-2.d-histogramas-de-y}

Creamos histogramas para R = 100 y R = 10\^{}4 utilizando 30 bines.

\paragraph{Distribución esperada:}\label{distribuciuxf3n-esperada-1}

Esperaría ver una distribución normal con media \(E(Y) = 9\) y varianza
\(V(Y) = 27/15\), especialmente para tamaños grandes de muestra. Con
R=100, el histograma debería comenzar a parecerse a una distribución
normal, y para R = 10\^{}4, la aproximación debería ser aún mejor.

\includegraphics{TP2_files/figure-latex/unnamed-chunk-9-1.pdf}

\paragraph{Discusión:}\label{discusiuxf3n-2}

\begin{itemize}
\tightlist
\item
  Para R=100: Es posible que la distribución de Y aún no se parezca
  completamente a una distribución normal, pero debería comenzar a
  mostrar una forma más aproximada a una campana en comparación con una
  distribución uniforme.
\item
  Para R=10\^{}4: Con un tamaño de muestra grande, la distribución de Y
  debería aproximarse más a una normal con media 9 y varianza 27/15.
  Esto es consistente con el Teorema Central del Límite, que establece
  que la media de muestras grandes de una variable aleatoria se
  distribuye aproximadamente normalmente, independientemente de la
  distribución original de la variable.
\end{itemize}

Efecto de variar R: A medida que aumenta R, la distribución de Y se
aproxima más a una normal con la media y varianza teóricas. Con un
tamaño de muestra pequeño, el histograma puede mostrar más variabilidad
y desviarse más de la distribución normal teórica.

\subsection{Ejercicio 3 - Teorema Central del
Límite}\label{ejercicio-3---teorema-central-del-luxedmite}

\subsubsection{Ejercicio 3.a: Histogramas de
X}\label{ejercicio-3.a-histogramas-de-x}

\includegraphics{TP2_files/figure-latex/unnamed-chunk-10-1.pdf}

\subsubsection{Ejercicio 3.b:
Histogramas}\label{ejercicio-3.b-histogramas}

\begin{Shaded}
\begin{Highlighting}[]
\NormalTok{X\_dist }\OtherTok{\textless{}{-}} \ControlFlowTok{function}\NormalTok{(R, N)\{}
\NormalTok{  Y }\OtherTok{\textless{}{-}} \FunctionTok{numeric}\NormalTok{(R)}
  \ControlFlowTok{for}\NormalTok{ (i }\ControlFlowTok{in} \DecValTok{1}\SpecialCharTok{:}\NormalTok{R)\{}
\NormalTok{    X\_15 }\OtherTok{\textless{}{-}} \FunctionTok{runif}\NormalTok{(N, }\DecValTok{0}\NormalTok{, }\DecValTok{18}\NormalTok{)}
\NormalTok{    mX\_15 }\OtherTok{\textless{}{-}} \FunctionTok{mean}\NormalTok{(X\_15)}
\NormalTok{    Y[i] }\OtherTok{\textless{}{-}}\NormalTok{ mX\_15}
\NormalTok{  \}}
  \FunctionTok{return}\NormalTok{(Y)}
\NormalTok{\}}

\NormalTok{color\_vec }\OtherTok{\textless{}{-}} \FunctionTok{c}\NormalTok{(}\StringTok{"\#689C6E"}\NormalTok{,}\StringTok{"\#73AD9C"}\NormalTok{,}\StringTok{"\#6F8AA6"}\NormalTok{,}\StringTok{"\#D2691E"}\NormalTok{)}

\NormalTok{R\_vec }\OtherTok{\textless{}{-}} \FunctionTok{c}\NormalTok{(}\DecValTok{10}\SpecialCharTok{\^{}}\DecValTok{2}\NormalTok{, }\DecValTok{10}\SpecialCharTok{\^{}}\DecValTok{6}\NormalTok{)}
\NormalTok{N\_vec }\OtherTok{\textless{}{-}} \FunctionTok{c}\NormalTok{(}\DecValTok{1}\NormalTok{, }\DecValTok{2}\NormalTok{, }\DecValTok{5}\NormalTok{, }\DecValTok{15}\NormalTok{)}

\FunctionTok{par}\NormalTok{(}\AttributeTok{mfrow =} \FunctionTok{c}\NormalTok{(}\DecValTok{4}\NormalTok{, }\DecValTok{2}\NormalTok{),}
    \AttributeTok{mar =} \FunctionTok{c}\NormalTok{(}\DecValTok{1}\NormalTok{, }\DecValTok{1}\NormalTok{, }\DecValTok{1}\NormalTok{, }\DecValTok{1}\NormalTok{),}
    \AttributeTok{oma =} \FunctionTok{c}\NormalTok{(}\DecValTok{1}\NormalTok{, }\DecValTok{1}\NormalTok{, }\FloatTok{2.5}\NormalTok{, }\DecValTok{1}\NormalTok{),}
    \AttributeTok{cex.main =} \FloatTok{1.0}\NormalTok{,}
    \AttributeTok{cex.lab =} \FloatTok{1.2}\NormalTok{,}
    \AttributeTok{cex.axis =} \FloatTok{1.1}\NormalTok{,}
    \AttributeTok{las =} \DecValTok{1}\NormalTok{,}
    \AttributeTok{mgp =} \FunctionTok{c}\NormalTok{(}\DecValTok{3}\NormalTok{, }\FloatTok{0.5}\NormalTok{, }\DecValTok{0}\NormalTok{))}

\ControlFlowTok{for}\NormalTok{ (irow }\ControlFlowTok{in} \DecValTok{1}\SpecialCharTok{:}\DecValTok{4}\NormalTok{) \{}
\NormalTok{  color }\OtherTok{\textless{}{-}}\NormalTok{ color\_vec[irow]}
\NormalTok{  N }\OtherTok{\textless{}{-}}\NormalTok{ N\_vec[irow]}
  \ControlFlowTok{for}\NormalTok{ (icol }\ControlFlowTok{in} \DecValTok{1}\SpecialCharTok{:}\DecValTok{2}\NormalTok{) \{}
\NormalTok{    R }\OtherTok{\textless{}{-}}\NormalTok{ R\_vec[icol]}
\NormalTok{    X }\OtherTok{\textless{}{-}} \FunctionTok{X\_dist}\NormalTok{(}\AttributeTok{R=}\NormalTok{R, }\AttributeTok{N =}\NormalTok{ N)}
    \FunctionTok{hist}\NormalTok{(X, }\AttributeTok{col =}\NormalTok{ color,}
         \AttributeTok{main =} \FunctionTok{paste}\NormalTok{(}\StringTok{"R ="}\NormalTok{, R), }\AttributeTok{freq =} \ConstantTok{FALSE}\NormalTok{)}
\NormalTok{  \}}
\NormalTok{\}}
\end{Highlighting}
\end{Shaded}

\includegraphics{TP2_files/figure-latex/unnamed-chunk-11-1.pdf}

\paragraph{Discusión:}\label{discusiuxf3n-3}

\begin{itemize}
\tightlist
\item
  Cambios al variar \(n\): a medida que \(n\) aumenta, la forma de la
  distribución de las muestras se va acercando más a una distribución
  normal. Esto es consistente con el Teorema Central del Límite, que
  establece que la suma de un número suficiente de variables aleatorias
  independientes e idénticamente distribuidas tenderá hacia una
  distribución normal, independientemente de la distribución original de
  las variables.

  \begin{itemize}
  \tightlist
  \item
    Para \(n = 1\), la distribución sigue siendo uniforme, ya que cada
    muestra es una realización directa de \(X\).
  \item
    A medida que aumentas \(n\), la dispersión de los datos disminuye a
    medida que \(n\) aumenta. Esto se debe a que la media de un mayor
    número de muestras es menos variable que la media de un número menor
    de muestras.
  \end{itemize}
\item
  Cambios al variar \(R\):

  \begin{itemize}
  \tightlist
  \item
    Para \(R = 10^2\), los histogramas muestran más variabilidad debido
    a que se toman menos muestras, lo que introduce más ruido en la
    estimación de la densidad.
  \item
    Para \(R = 10^6\), la distribución estimada se aproxima mejor a la
    forma teórica de la distribución, ya que un mayor número de
    repeticiones reduce la variabilidad en la estimación.
  \end{itemize}
\end{itemize}

En resumen:

\begin{itemize}
\tightlist
\item
  Al variar \(n\): Se observa una transición de una distribución
  uniforme hacia una distribución normal al aumentar \(n\). Este cambio
  ilustra la aplicación del Teorema Central del Límite, donde la suma (o
  promedio) de un número creciente de variables aleatorias idénticamente
  distribuidas tiende a una distribución normal.
\item
  Al variar \(R\): La forma general de los histogramas no cambia
  significativamente con \(R\); lo que cambia es la precisión con la que
  esa forma se estima. Con un \(R\) mayor, se obtiene una mejor
  aproximación de la forma verdadera de la distribución subyacente.
  \(R\) no cambia la naturaleza de la distribución, solo mejora la
  precisión con la que se representa.
\end{itemize}

\end{document}
